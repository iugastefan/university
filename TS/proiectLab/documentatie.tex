\documentclass{article}
\usepackage{amsmath}
\usepackage{eurosym}
\usepackage{listings}
\usepackage[dvipsnames]{xcolor}
\lstdefinestyle{mystyle}{
	backgroundcolor=\color{lightgray},   
	commentstyle=\color{cyan},
	keywordstyle=\color{magenta},
	numberstyle=\tiny\color{black},
	stringstyle=\color{cyan},
	basicstyle=\ttfamily\footnotesize,
	breakatwhitespace=false,         
	breaklines=true,                 
	captionpos=b,                    
	keepspaces=true,                 
	numbers=left,                    
	numbersep=5pt,                  
	showspaces=false,                
	showstringspaces=false,
	showtabs=false,                  
	tabsize=2
}

\lstset{style=mystyle}

\title{Proiect tehnici de simulare}
\author{\c{S}tefan Iuga \and Iancu Petcu \and \c{S}tefan Pogonaru}
\begin{document}
	\maketitle
	\section{Descrierea problemei}
	Propunem analiza activit\u{a}\c{t}ii unei firme ce ofer\u{a} poli\c{t}e
	de asigurare pentru echipamente electronice.

	Scopul simul\u{a}rii este studierea eficien\c{t}ei de v\^{a}nzare a
	acestei poli\c{t}e \^{i}n decursul a T de zile.

	\section{Obiectivele simul\u{a}rii}
	\begin{enumerate}
		\item Estimarea probabilit\u{a}\c{t}ii ca firma s\u{a} nu fie
			ruinat\u{a} p\^{a}n\u{a} la momementul T.
		\item Estimarea unui capital minim necesar ca probabilitatea
			estimata de ruin\u{a} p\^{a}n\u{a} la momentul T s\u{a}
			fie mai mic\u{a} de 80\%.
	\end{enumerate}

	\section{Conven\c{t}ii}
	\begin{itemize}
		\item  Pornim cu un num\u{a}r ini\c{t}ial de $n_0$
			clien\c{t}i \c{s}i un capital $a_0$.
		\item Clien\c{t}ii fac cereri de desp\u{a}gubire conform unui
			proces Poisson omogen de rat\u{a}
			$\lambda$.
		\item Valoarea fiec\u{a}rei desp\u{a}gubiri solicitate este o
			variabil\u{a} aleatoare cu func\c{t}ia de 
			reparti\c{t}ie $F$.
		\item Poten\c{t}ialii noi clien\c{t}i ai firmei semneaz\u{a}
			contracte de asigurare conform unui proces Poisson
			omogen de rat\u{a} $\nu$.
		\item Actualii clien\c{t}i ai firmei r\u{a}m\^{a}n fideli
			firmei pentru un timp aleator ce corespunde
			reparti\c{t}iei exponen\c{t}iale de parametru
			$\mu$.
		\item To\c{t}i clien\c{t}ii firmei pl\u{a}tesc o sum\u{a} fix\u{a}
			$c$ per unitate de timp.
	\end{itemize}
	\section{Datele problemei}
	\begin{itemize}
		\item $\lambda=8/zi$
		\item $F(x) =
			\begin{cases}
				0 & x<0\\
				\frac{x^2+x}{2} & x \in [0,1) \\
				1 & x\geq1
			\end{cases}
			$
		\item $\nu = 18/zi$
		\item $\mu = 0.2 (zile)$
		\item $c = 2 $\euro$/zi$
		\item $T = 365 zile$
		\item $a_0 = 50000$\euro
		\item $n_0=12$
	\end{itemize}
	\section{Cod sursa}
	\lstinputlisting[language=R]{ts.R}
	
	\section{Rezultate}
	Am ob\c{t}inut o probabilitate 0 de faliment cu capitalul ini\c{t}ial $50000$\euro.
	
	Capitalul necesar pentru ca rata de faliment s\u{a} fie mai mic\u{a} de 80\% este $42500$\euro.
\end{document}

